\newpage
\section{Posicionamento final}
Na sequência da investigação técnica e da análise de mercado realizadas, conclui-se que a implementação de uma API robusta e segura para a gestão de cold wallets, no contexto do Porto by Anchorage Digital, é viável exclusivamente do ponto de vista técnico. Contudo, a evidência recolhida indica que a solução proposta não se encontra alinhada com as necessidades, prioridades e expectativas atuais do mercado institucional de ativos digitais, no qual a Anchorage Digital se posiciona.

A avaliação das necessidades dos clientes institucionais, atuais e potenciais, demonstra que estes tendem a privilegiar soluções que assegurem uma integração fluida com infraestruturas existentes, aliada a capacidades avançadas de gestão de chaves criptográficas, definição de políticas de segurança e controlo de risco. Embora a disponibilização de APIs para cold wallets seja tecnicamente exequível, não se revela, neste momento, um requisito crítico para este segmento de clientes. Em contrapartida, existe uma clara preferência por soluções de custódia integradas, que ofereçam segurança end-to-end, simplicidade operacional e redução da complexidade associada à gestão direta de componentes de baixo nível.

\subsection{Síntese das razões para não avançar com a implementação da API}

\begin{itemize}
    \item \textbf{Baixa prioridade comercial:} ausência de procura explícita por APIs de cold wallets por parte do segmento institucional.

    \item \textbf{Preferência por soluções integradas:} maior valorização de plataformas completas de custódia, em detrimento de componentes isolados expostos via API.

    \item \textbf{Elevada maturidade do mercado:} presença de concorrentes com soluções já consolidadas, reduzindo o potencial de diferenciação.

    \item \textbf{Relação custo-benefício desfavorável:} esforço de implementação, manutenção e certificação regulatória elevado face ao retorno esperado.

    \item \textbf{Risco de dispersão estratégica:} potencial desvio de foco relativamente a áreas com maior impacto no produto e no negócio.
\end{itemize}