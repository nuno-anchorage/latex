\newpage
\section{Introdução}
A exposição de produtos através de Interfaces de Programação de Aplicações (APIs) constitui uma prática consolidada no desenvolvimento de soluções digitais escaláveis e interoperáveis. Esta abordagem permite que componentes de software distintos comuniquem entre si de forma programática, assegurando a reutilização de serviços, a normalização de acessos e a integração eficiente com sistemas externos~\cite{fielding2000, papazoglou2008}.

Ao disponibilizar funcionalidades de um produto via API, torna-se possível promover a integração com parceiros, clientes ou sistemas internos, num modelo desacoplado e controlado. Esta estratégia não só reduz a dependência de interfaces manuais como também facilita a automação de processos e a criação de novos canais digitais de distribuição~\cite{rodrigues2020}.

Além disso, o uso de APIs facilita a adoção de arquiteturas modernas, como microserviços, e potencia a escalabilidade horizontal dos sistemas~\cite{newman2015}. Do ponto de vista económico e estratégico, a exposição via API permite a criação de ecossistemas digitais, nos quais terceiros podem desenvolver aplicações complementares, ampliando o valor do produto principal e promovendo a inovação aberta~\cite{ibm_api_economy, o2016}.

A crescente adoção de ativos digitais e o desenvolvimento da Web3 têm impulsionado a necessidade de soluções de custódia que sejam simultaneamente seguras, interoperáveis e escaláveis. Neste contexto, a exposição programática de funcionalidades de crypto-wallets através de APIs torna-se essencial para assegurar a integração fluida com plataformas descentralizadas, serviços financeiros digitais e mecanismos de identidade soberana~\cite{antonopoulos2017, vanwirdum2021}.

Ao expor uma crypto-wallet por meio de uma API, é possível desacoplar a camada de apresentação da lógica de custódia, permitindo que múltiplos clientes, aplicações móveis ou sistemas de terceiros interajam com a carteira de forma programática e segura~\cite{gudgeon2020}. Esta abordagem promove a reutilização de serviços críticos, como a assinatura de transações, a gestão de chaves, ou o controlo de permissões de acesso, garantindo simultaneamente a auditabilidade e rastreabilidade das operações~\cite{bonneau2015}.

Além disso, a disponibilização de uma carteira através de API permite acelerar a inovação no ecossistema Web3, uma vez que facilita a criação de integrações com bolsas descentralizadas (DEXs), plataformas de empréstimo, serviços de staking, e mecanismos de autenticação baseados em blockchain. Tal como em outras áreas da computação distribuída, a API atua como um contrato semântico e técnico entre fornecedores e consumidores de serviços, favorecendo a escalabilidade e a evolução modular da solução~\cite{fielding2000, papazoglou2008}.

A exposição programática de crypto-wallets é particularmente relevante em contextos institucionais, nos quais se impõe o cumprimento de requisitos regulatórios, separação de funções e controlos de acesso granulares. Nestes casos, a API funciona também como uma camada de abstração para garantir conformidade com normas de segurança, como o uso de HSMs (Hardware Security Modules) ou técnicas de MPC (Multi-Party Computation)~\cite{boneh2019, anchoragempc2023}.
