\newpage
\thispagestyle{otherpages}
\section{Introdução}
A API da Anchorage fornece uma interface segura e robusta para instituições interagirem com a plataforma de custódia de ativos digitais da Anchorage. Esta secção descreve os componentes principais, medidas de segurança e aspectos operacionais da API.

\subsection{Sistema de Autenticação}
\subsubsection{Componentes Principais}
\begin{itemize}
\item Chaves API (Tokens Bearer)
\item Grupos de Permissões
\item Assinaturas Ed25519
\end{itemize}

\subsubsection{Fluxo de Autenticação}
Cada pedido à API passa por um processo de autenticação em várias etapas:
\begin{enumerate}
\item Verificação da validade da chave da API
\item Confirmação das permissões associadas à chave
\item Para pedidos específicos, validação das assinaturas Ed25519
\end{enumerate}

\subsubsection{Requisitos de Assinatura}
Para pedidos da API v2 que requerem assinaturas:
\begin{itemize}
\item A Anchorage mantém a chave pública no seu sistema
\item Os clientes devem armazenar e gerir de forma segura a sua chave privada
\item Cada pedido é validado assimetricamente usando a assinatura do cliente
\end{itemize}

\subsection{Gestão de Permissões}
\subsubsection{Grupos de Permissões}
Os grupos de permissões são fundamentais para o controlo de acesso à API:
\begin{itemize}
\item Devem ser estabelecidos antes da criação da chave da API
\item Definem regras específicas de acesso para recursos organizacionais
\item Controlam permissões granulares para diferentes endpoints da API
\end{itemize}

\subsubsection{Requisitos de Quórum}
A modificação dos grupos de permissões requer:
\begin{itemize}
\item Múltiplos utilizadores autorizados para aprovar alterações
\item Endosso baseado em quórum para operações críticas
\item Verificação sistemática da cadeia de aprovação
\end{itemize}

\subsection{Administração de Chaves API}
\subsubsection{Gestão de Chaves}
Os controlos administrativos para chaves API incluem:
\begin{itemize}
\item Criação restrita apenas a administradores autorizados
\item Capacidade de revogar chaves imediatamente quando necessário
\item Monitorização e rastreamento do uso das chaves
\end{itemize}

\subsubsection{Responsabilidades de Segurança}
Considerações Importantes de Segurança:
\begin{itemize}
\item As organizações são totalmente responsáveis pela distribuição das chaves API
\item Todas as ações realizadas com chaves API válidas são consideradas autorizadas
\item A Anchorage não pode reverter operações executadas via API
\item Recomenda-se a revogação imediata da chave em caso de suspeita de comprometimento
\end{itemize}

\subsection*{Migração de APIs para o Porto}

A migração das APIs atuais para o Porto apresenta um desafio complexo e multifacetado que exige consideração meticulosa, planeamento estratégico abrangente e execução precisa. Na base destes desafios reside uma transformação arquitetural fundamental e de longo alcance: o Porto opera como um sistema sofisticado de carteira de autocustódia onde as chaves privadas são mantidas de forma segura pelos clientes e partilhadas eficientemente entre os seus utilizadores designados, representando uma mudança arquitetural significativa da infraestrutura existente da Anchorage Digital, onde a empresa mantém o controlo centralizado sobre as chaves privadas.

De uma perspetiva técnica, inúmeras funcionalidades críticas das APIs enfrentam atualmente desafios substanciais de compatibilidade com o ambiente Porto. Operações essenciais envolvendo a criação de carteiras, mecanismos de transferência e funcionalidades de retenção ainda não alcançaram o estado operacional completo dentro do ecossistema Porto. Além disso, a infraestrutura da API de transações brutas requer desenvolvimento extenso para atingir o estado de produção, particularmente no que diz respeito à integração de serviços essenciais de terceiros como o Uniswap.

Para abordar e resolver estes desafios, a equipa de desenvolvimento projetou uma estratégia de implementação estruturada e faseada. Esta abordagem metódica começa com a implementação de APIs de Leitura, progride para APIs de Escrita incorporando mecanismos robustos de aprovação por quórum, e conclui com uma solução abrangente que opera sem requisitos de quórum.

Os requisitos de implementação incluem uma extensa matriz de componentes interligados. Um aspeto crítico envolve a reativação da funcionalidade da API em múltiplas plataformas, focando-se tanto na interface do Painel do Cliente como nas aplicações móveis iOS. A equipa de desenvolvimento deve executar testes abrangentes de vários \textit{endpoints} específicos, incluindo sistemas de gestão de endereços, operações de carteira (excluindo funcionalidade de criação), capacidades de cofre, mecanismos de negociação, procedimentos de gestão de tipos de ativos e protocolos de gestão de chaves API.

O impacto desta migração estende-se além das considerações técnicas. O esforço de desenvolvimento irá melhorar as capacidades da Plataforma de Gestão de Ativos Digitais através da integração com sistemas e serviços externos. Embora certos clientes tenham implementado soluções provisórias, como a monitorização direta da atividade \textit{blockchain}, a implementação das APIs Porto completas fornecerá uma solução otimizada e eficiente para atender às suas necessidades operacionais e de negócio.

Em conclusão, os desafios na migração de APIs para a plataforma Porto, embora substanciais e complexos, representam passos necessários na evolução das capacidades e ofertas de serviço da plataforma. A abordagem de implementação estruturada, combinada com protocolos de teste abrangentes e procedimentos de validação, garantirá a entrega de uma infraestrutura de API eficiente que atende aos requisitos do sistema avançado de carteira de autocustódia do Porto, mantendo a segurança e desempenho ideais.